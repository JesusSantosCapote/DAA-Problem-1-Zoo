\documentclass[article]{llncs}
%
\usepackage[utf8]{inputenc}
\usepackage[spanish]{babel}
\usepackage{graphicx}
% Used for displaying a sample figure. If possible, figure files should
% be included in EPS format.
%
% If you use the hyperref package, please uncomment the following line
% to display URLs in blue roman font according to Springer's eBook style:
% \renewcommand\UrlFont{\color{blue}\rmfamily}


\begin{document}
%
\title{Dise\~{n}o y An\'alisis de Algoritmos. Problema 1: El Zool\'ogico}
%
%\titlerunning{Abbreviated paper title}
% If the paper title is too long for the running head, you can set
% an abbreviated paper title here
%
\author{Jes\'us Santos Capote y Kenny Villalobos Morales}
%
\institute{Facultad de Matemática y Computación, Universidad de La Habana, La Habana, Cuba }
%
\maketitle              % typeset the header of the contribution
%
\section{Definici\'on del Problema}

En un zoológico un poco especial, se separa a los animales en dos 
hábitats diferentes de manera general. El hábitat para la reproducción 
solo acepta animales de la misma especie, mientras que el hábitat para 
la maduración admite animales de distintas especies, pero no de géneros 
distintos. El zoológico está pasando por una remodelación, ya que va 
a recibir n especies distintas de animales, cada una con un número 'a' 
de hembras y 'b' de machos, que puede ser distinto entre cada especie. 
En la remodelación se está pensando en construir salas de exhibición de 
los animales. Cada una se construirá para ser un hábitat de reproducción
 o de maduración, y cada sala podrá soportar un máximo de k animales, 
 lo cual es igual para todas las salas. Como el zoológico necesita ser 
 rentable y cada sala se cobra por separado, se quiere construir el 
 máximo número posible de salas que cumplan con los requerimientos 
 planteados y que, además, estén llenas de animales para el disfrute 
 de los visitantes. ¿Cuántas salas deberá construir el zoológico?

\section{Modelaci\'on del Problema}

Se tiene una matriz de dimensi\'on $2*n$, donde cada casilla contiene n\'umeros 
enteros. El problema consiste en hallar la cantidad m\'axima de grupos de 
tama\~{n}o $k$ que se pueden formar, donde cada elemento de un grupo es 
una unidad perteneciente a una casilla de la matriz. Adem\'as, todos los 
elementos de un grupo deben pertenecer a casillas de una misma fila o de 
una misma columna de la la matriz.

\section{Primera Aproximaci\'on}

Como primera soluci\'on se propone aplanar la matriz de entrada para 
obtener una lista de $T$ elementos, siendo $T$ la suma de todos los elementos de la matriz,
 donde en cada posici\'on hay un animal. 
Los animales son modelados como una clase \emph{Animal} que tiene dos 
atributos: \emph{gender}, fila a la que pertenece en la matriz (g\'enero),  
y \emph{specie}, columna a la que pertenece en la matriz (especie). Luego 
generar todas las permutaciones de la matriz aplanada y por cada una 
dividirla mediante \'indices en porciones de tamaño $k$, contar cuantas 
de estas porciones son grupos v\'alidos y actualizar, de ser necesario, el 
máximo;   

\subsection{Optimalidad}
Sea $S = [a_1, a_2, \dots, a_T]$ una distribución de los animales de la 
matriz aplanada, tal que la cantidad de porciones de tamaño $k$ que se pueden 
formar a partir de ella, que cumplen las restricciones del problema, sea máxima. 
Esta distribución $S$ es una permutación de los elementos de la matriz aplanada. 
El algoritmo propuesto revisa todas las permutaciones de la matriz aplanada y 
calcula la cantidad de grupos factibles, por tanto, revisa a $S$. Luego 
el algoritmo encuentra la respuesta óptima.

\subsection{Complejidad Temporal}

Aplanar la matriz tiene costo $O(T)$ donde T es la cantidad 
de animales. Computar todas las permutaciones de una lista 
de tamaño T tiene un costo $O(T!)$. La comprobación 
de cuántos grupos factibles contiene una permutación de T 
elementos tiene complejidad $O(T)$ y esta comprobación se realiza para cada permutación.
Luego por el teorema de la multiplicación contar cuantos grupos factibles tiene cada permutación 
tiene complejidad $O(T*T!)$. Luego el algoritmo 
tiene complejidad $O(T + T*T! + T)$ y por el teorema de la 
suma, la complejidad final es $O(T*T!)$.

\section{Segunda Aproximaci\'on}

Demostremos que existe un \'optimo que solo usa un grupo columna a lo sumo 
por columna:

Sea $a_i$ y $b_i$ los elementos de la columna $i$. Sea $a_i + b_i = c$.

$$c = p_c * k + r_c, \hspace{0,2cm} r_c < k$$

$p_c$, es la cantidad de grupos columna que se pueden formar en la columna $i$.

$$a_i = p_a * k + r_a, \hspace{0,2cm} r_a < k$$

$p_a$, es la cantidad de grupos fila que se pueden formar solo con los elementos de $a_i$.

$$b_i = p_b * k + r_b, \hspace{0,2cm} r_b < k$$

$p_b$, es la cantidad de grupos fila que se pueden formar solo con los elementos de $b_i$. Por tanto:

$$c = a_i + b_i = p_a * k + p_b * k + r_a + r_b, \hspace{0,2cm} r_a < k, r_b < k$$

$$c = (p_a + p_b)*k + r_a + r_b$$

con $r_a + r_b < 2k$.

Si $r_a + r_b < k$ entonces por el algoritmo de divisi\'on, la cantidad de grupos que 
se pueden formar en la columna $i$, $\lfloor\frac{c}{k}\rfloor$, es igual a la suma 
de $p_a$ (cantidad de grupos que se pueden formar solo con los elementos de $a_i$) mas 
$p_b$ (cantidad de grupos que se pueden formar solo con los elementos de $b_i$). Es decir, para este caso,
con formar solo grupos filas con los elementos de $a_i$ y $b_i$ se iguala la cantidad 
de grupos columnas que se pueden formar en la columna $i$. 

Si $k \leq r_a + r_b < 2k$ entonces $r_a + r_b = r' + k$ con $r' < k$, luego:

$$c = (p_a + p_b)*k + r' + k$$
$$c = (p_a + p_b + 1)*k + r', \hspace{0,2cm} r' < k$$

Esto significa que, por el algoritmo de divisi\'on, la cantidad de grupos que 
se pueden formar en la columna $i$, $\lfloor\frac{c}{k}\rfloor$, es igual a la suma 
de $p_a$ (cantidad de grupos que se pueden formar solo con los elementos de $a_i$) mas 
$p_b$ (cantidad de grupos que se pueden formar solo con los elementos de $b_i$) mas uno. 
Es decir, la cantdad de grupos $p_c$ que se pueden formar en una columna es igual a la cantidad $p_a + p_b + 1$ 
a lo sumo, con formar solo grupo filas con los elementos de $a_i$ y $b_i$
obtengo la cantidad m\'axima de grupos menos uno.

Por lo dicho anteriormente existe un \'optimo que tiene a lo sumo un grupo con elementos pertenecientes a una misma columna $i$
para cada $i$.

\subsection{Explicaci\'on del Algoritmo}

Partiendo de que solo se necesita tomar un grupo columna por columna a lo sumo y 
tomar el resto como grupos fila, el algoritmo para cada forma de seleccionar 
columnas prueba todas las formas de crear un grupo columna en cada columna seleccionada. 
Luego de tener un conjunto de columnas seleccionadas y una forma de crear grupos columna fijada se 
calcula la cantidad de grupos filas que se pueden formar y se suma a la cantidad de grupos columnas 
que se crearon. El retorno del algoritmo sera el mayor n\'umero computado de esta forma, que corresponde
al n\'umero máximo de grupos que se pueden crear cumpliendo las restricciones del problema.

\subsection{Complejidad Temporal}



\section{Tercera Aproximaci\'on}

Sea $M = \sum_{i=0}^{n}matriz[0, i]$, $F = \sum_{i=0}^{n}matriz[1, i]$ y 
$T = M + F$.

Sabemos que la cantidad máxima de grupos que se pueden formar es $\lfloor \frac{T}{k} \rfloor$.
Para toda posible entrada de datos del problema, la solución es mayor o igual que 
$\lfloor \frac{T}{k} \rfloor - 1$. Demostremos esto: 

%

Cumpliendo con las restricciones del problema, siempre es posible formar 
$\lfloor \frac{M}{k}\rfloor=p_1  + \lfloor \frac{F}{k}\rfloor=p_2 $ grupos factibles, 
donde cada unidad de los grupos pertenece a una misma fila, quedando $r_1$ y 
$r_2$ elementos residuales en cada fila respectivamente. Luego:

$$M = p_1 * k + r_1, \hspace{0,2 cm} r_1 < k$$
$$F = p_2 * k + r_2, \hspace{0,2 cm} r_2 < k$$
$$T = M + F = p_1 * k + p_2*k + r_1 + r_2$$

Como $r_1, r_2 < k$, entonces $r_1 + r_2 < 2k$. Luego, si $r_1 + r_2 < k$, tomando 
$r_3 = r_1 + r_2$, se cumple que:
$$T = (p_1 + p_2)k + r_3, \hspace{0,2 cm} r_3 < k$$
Luego $\lfloor \frac{T}{k}\rfloor = p_1 + p_2 = \lfloor \frac{M}{k}\rfloor  + \lfloor \frac{F}{k}\rfloor$

\vspace{0,2 cm}

Si $r_1 + r_2$ no es menor que $k$ entonces $r_1 + r_2 \geq k$, entonces $r_1 + r_2 = r_3 + k$
con $r_3 < k$. Luego se cumple que:

$$T = (p_1 + p_2)k + k + r_3 = (p_1 + p_2 + 1)k + r_3$$
Luego $\lfloor \frac{T}{k}\rfloor = p_1 + p_2 + 1$ entonces $\lfloor \frac{T}{k}\rfloor -1 = p_1 + p_2 = \lfloor \frac{M}{k}\rfloor  + \lfloor \frac{F}{k}\rfloor$

\vspace{0,2 cm}

Para el caso en que $r_1 + r_2 < k$, como $\lfloor \frac{T}{k}\rfloor = p_1 + p_2$ esta solución es factible 
y óptima.

Para el caso en que $r_1 + r_2 \geq k$, $p_1 + p_2 = \lfloor \frac{T}{k}\rfloor -1$ es 
la soluci\'on \'optima si no es posible crear grupos con elementos pertenecientes a una misma columna.

En otro caso, en el que existe $i$ tal que $a_i + b_i \geq k$, $p_1 + p_2$
no se garantiza, de momento, que sea una solución \'optima. Se podr\'ia 
buscar $i$ tal que $min(a_i, r_1) + min(b_i, r_2) \geq k$, para este caso 
bastar\'ia con asignar el resto de cada fila a $a_i$ y $b_i$ respectivamente
con ello se formar\'ia un grupo m\'as con elementos pertenecientes a una misma columna, 
sin disminuir la cantidad $p_1 + p_2$ de grupos que se pueden formar, logrando 
crear $p_1 + p_2 + 1$ grupos que es un resultado \'optimo.

Pero de no existir dicho $i$ no se garantiza a\'un que $p_1 + p_2$ sea 
un resultado \'optimo, cabe la posibilidad de que dejando de crear 
$m$ grupos con elementos pertenecientes a una misma fila se pudiesen 
formar $m+1$ grupos con elementos pertenecientes a una misma columna.
N\'otese que para las columnas $i$ tal que $a_i + b_i < k$ los elementos 
de $a_i$ y $b_i$ solo pueden ser utilizados para formar grupos de elementos
pertenecientes a una misma fila.

Esto se puede ver como el siguiente subproblema:

Se tiene una submatriz de la matriz $A$ de entrada donde solo est\'an 
las columnas $i$ pertenecientes a $A$ tal que $A[0, i] + A[1, i] \geq k$.
Sea $r_1 = M - p_1*k$ y $r_2 = F - p_2*k$, tal que $r_1, r_2 < k$ y 
$r_1 + r_2 \geq k$ ($p_1 + p_2 = \lfloor\frac{T}{k}\rfloor - 1$). ¿Es posible encontrar un conjunto de columnas $S$ de 
tamaño $m$ tal que pueda formar un grupo con elementos pertenecientes a 
una misma columna por cada columna en $S$, dejando de hacer $m-1$ grupos 
de elementos pertenecientes a una misma fila? ($\lfloor\frac{T}{k}\rfloor = p_1 + p_2 + m - (m-1)$)

Anteriormente se demostr\'o que existe un \'optimo que solo usa un grupo columna a lo sumo 
por columna. Si existiese una solución $S$ de este subproblema se debe cumplir que:

%TODO: Especificar la composici'on del conjunto S de columnas son las columnas con lo que se utiliza de cada una de ellas para hacer el grupo en dicha columna

Siendo:
$$M' = M - \sum S[0,i]$$
$$F' = F - \sum S[1, i]$$

Entonces $M' \% k + F' \% k < k$ pues: 
$$T = |S| * k + \lfloor\frac{M'}{k}\rfloor*k + M'\%k + \lfloor\frac{F'}{k}\rfloor*k + F' \% k$$

$$T = (|S| + \lfloor\frac{M'}{k}\rfloor + \lfloor\frac{F'}{k}\rfloor)*k + M'\%k + F'\%k$$

Por lo que por el algoritmo de la división: $|S| + \lfloor\frac{M'}{k}\rfloor + \lfloor\frac{F'}{k}\rfloor = \lfloor\frac{T}{k}\rfloor$

$$T = \lfloor\frac{T}{k}\rfloor*k + T \% k$$

Nótese que si $M' \% k + F' \% k < k$, entonces, además, $M' \% k + F' \% k = T \% k$

Por tanto, si existe una solución $S$, $M' \% k + F' \% k = T \% k$.

Demostremos que $M' \% k + F' \% k = T \% k \Leftrightarrow M' \% k \leq T \% k$:

$\Rightarrow$

Supongamos con el fin de llegar a un absurdo que $M' \% k > T\%k$,
entonces $M'\%k + F'\%k > T\%k$, contradicci\'on, pues 
$M' \% k + F' \% k = T \% k$. Luego $M' \% k \leq T \% k$

$\Leftarrow$

$$T = (|S| + \lfloor\frac{M'}{k}\rfloor + \lfloor\frac{F'}{k}\rfloor)*k + F'\%k + M'\% k$$

$$T\%k = ((|S| + \lfloor\frac{M'}{k}\rfloor + \lfloor\frac{F'}{k}\rfloor)*k + F'\%k + M'\% k)\%k$$

$$T\%k = (|S| + \lfloor\frac{M'}{k}\rfloor + \lfloor\frac{F'}{k}\rfloor)*k\%k + (F'\%k + M'\% k)\%k$$
$$T\%k = 0 + (F'\%k + M'\% k)\%k$$
$$T\%k = (F'\%k + M'\% k)\%k$$

Luego $T$ y $(F'\%k + M'\% k)$ son congruentes m\'odulo $k$.

Si $M' \% k \leq T \% k$ entonces $F' \%k = T\%k - M'\%k + x*k$.
Como $F' \% k < k$, entonces $x = 0$ y $F' \% k = T\%k - M'\%k$. Luego 
$M' \%k + F' \% k  = T \% k$

Por tanto si existiese una solución $S$ se debe cumplir que:

$$(M - \sum S[0, i])\% k \leq T\%k$$

$$M \%k - T\%k \leq (\sum S[0,i])\%k$$ 
y además como $(M - \sum S[0,i]) \geq 0$

$$M\%k \geq \sum S[0, i] \%k$$

entonces se cumple que: 

$$M\%k - T\% k \leq (\sum S[0,i])\%k \leq M \% k$$

En este punto el problema se ve reducido a determinar si existe $S$ tal que: $$M\%k - T\% k \leq (\sum S[0,i])\%k \leq M \% k$$.
Por tanto, si se pudiese calcular los valores de $(\sum S[0,i])\%k$ que son posibles dada la matriz, se pudiese saber si es posible armar $\lfloor\frac{T}{k}\rfloor$
grupos.

\subsection{Uso de Programación Dinámica}

Inicialmente nos interesa saber por cada columna $m_i$ de $M$, los posibles valores que se pueden utilizar de $m_i$ para formar un grupo en su columna. Llamemos a estos valores $m_{i,j}$.
Como sabemos que los posibles valores que puedo tomar $(\sum S[0,i])\%k$ son a lo sumo $k$ es posible representar estos valores a traves de un array $z$ de tamaño $k$. Donde la posición i-ésima de $z$ tiene
un 1 si es posible que $(\sum S[0,i])\%k$ tomo el valor $i$ y un 0 si no (en realidad solo nos interesan $k-1$ posibles valores que puede tomar $(\sum S[0,i])\%k$, pues que $(\sum S[0,i])\%k = 0$ nunca es solución, ya que $M\%k > T\%k$).

Sea $z_i$ el array que representa los posibles valores que puede tomar $(\sum S[0,i])\%k$ usando solo las primeras $i$ columnas de la submatriz,
a partir de $z_i$ es posible calcular $z_{i+1}$ de la siguiente manera:

$\bullet $ Todos los valores que son 1 en $z_i$ también los son en $z_{i+1}$, pues si era posible hacer dicho valor sin considerar la columna $i+1$, simplemente se puede lograr sin usarla.

$\bullet $ Todos los valores $m_{i+1,j}$ son 1 en $z_{i+1}$, puesto que estos son posible lograrse sin utilizar ninguna de las anteriores $i$ columnas.

$\bullet $ Por cada valor $x$ en $1$ en $z_i$, los valores $(x+m_{i+1,j}) \% k$ son $1$ en $z_{i+1}$ para todo $j$.

Para $z_1$ basta con hacer $1$ todos los valores $m_{1,j}$. Calculando los $z_i$ hasta el último de ellos, en este estarán marcados en 1 todos los valores de $(\sum S[0,i])\%k$ que son posibles dada la matriz.

Luego basta comprobar alguno de los posibles valores de $(\sum S[0,i])\%k$ pertenece al intervalo $$M\%k - T\% k \leq (\sum S[0,i])\%k \leq M \% k$$.

\subsection{Complejidad Temporal}

Calcular la cantidad de elementos pertenecientes a cada fila tiene un costo $O(n)$. Calcular los valores $m_{i,j}$ tiene un costo $O(n*k)$, pues para cada columna $n$ es posible que se usen a lo sumo $k$ posibles valores distintos.
Crear los arrays $z_i$ tiene un costo $O(n*k^2)$, pues por cada columna $z_i$ de las n columnas, es necesario verificar los posibles $k$ valores
de $(\sum S[0,i])\%k$ que se pudieron generar en $z_{i-1}$ y por cada uno de ellos comprobar que valores se pueden lograr sumando con los $k$
posibles valores de los $m_{i,j}$. Luego verificar si alguno de los valores posibles de $(\sum S[0,i])\%k$ pertenece al intervalo $[M\%k - T\% k , M \% k]$ tiene un costo $O(k)$.
Luego el algoritmo 
tiene complejidad $O(n + n*k + n*k^2 + k)$ y por el teorema de la 
suma, la complejidad final es $O(n*k^2)$.

\section{Generadores de Casos Prueba}

Se implementaron tres generadores de casos prueba. Su funcionamiento 
consiste en generar unos $n,k$ aleatoriamente, para luego generar 
los dos arrays de entrada. Sea $a_i$ y $b_i$ los elementos de los dos arrays
de entrada en la posici\'on $i$ respectivamente. Para generar ambos arrays, 
se generan de forma aleatoria $a_i$ y $b_i$ para cada $i$, tal que 
$a_i + b_i > 0$.
Lo que diferencia a los tres generadores son los rangos de generaci\'on para 
cada variable que se genere aleatoriamente.

\subsection{generator1}
Este generador est\'a pensado para las pruebas que se realicen con el algoritmo 
propuesto en la primera aproximaci\'on. Debido a la alta complejidad de 
este primer algoritmo, los casos que crea generator1, son relativamente peque\~{n}os.

\subsection{generator2}
Este generador est\'a pensado para las pruebas que se realicen con los algoritmos 
propuestos en la segunda y tercera aproximaci\'on. Genera casos m\'as grandes 
que el primer generador.

\subsection{generator3}
Este generador est\'a pensado para las pruebas que se realicen con el 
algoritmo de la tercera aproximaci\'on. Los casos que este generador 
crea son bastante grandes, tanto que provar los algoritmos de las primeras 
aproximaciones es impracticable.

\section{Tester}

El tester es una aplicaci\'on de consola interactiva. Donde el usuario 
debe escoger dos algoritmos para comparar sus resultados, un generador 
de casos prueba y una cantidad de casos prueba. Para cada caso prueba 
el tester compara la soluci\'on de los dos algoritmos seleccionados. 
En caso de que coincidan imprime un mensaje en color verde, en caso 
de que no coincidan imprime un mensaje en rojo.

Se aconseja realizar las pruebas de los algoritmos con generator1 debido 
a la rapidez con que los algoritmos propuestos resuelven los casos.

Se pueden comparar los resultados de los algoritmos de la segunda y 
tercera aproximaci\'on con generator2, aunque la prueba ser\'a m\'as 
lenta debido a la alta complejidad temporal de la segunda aproximaci\'on.

El generador generator3 est\'a dise\~{n}ado para probar la rapidez del 
algoritmo de la tercera aproximaci\'on, no para la comparaci\'on con los 
restantes algoritmos.

\subsection{Instrucciones}

Abrir una consola en la carpeta donde se encuentra el archivo tester.py. 
Ejecutar el comando python tester.py.

Si se quisiera no comparar dos algoritmos, sino probar la rapidez contra un 
generador determinado, seleccione el algoritmo dos veces.

\end{document}