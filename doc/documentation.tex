\documentclass[article]{llncs}
%
\usepackage[utf8]{inputenc}
\usepackage[spanish]{babel}
\usepackage{graphicx}
% Used for displaying a sample figure. If possible, figure files should
% be included in EPS format.
%
% If you use the hyperref package, please uncomment the following line
% to display URLs in blue roman font according to Springer's eBook style:
% \renewcommand\UrlFont{\color{blue}\rmfamily}


\begin{document}
%
\title{Dise\~{n}o y An\'alisis de Algoritmos. Problema 1: El Zool\'ogico}
%
%\titlerunning{Abbreviated paper title}
% If the paper title is too long for the running head, you can set
% an abbreviated paper title here
%
\author{Jes\'us Santos Capote y Kenny Villalobos Morales}
%
\institute{Facultad de Matemática y Computación, Universidad de La Habana, La Habana, Cuba }
%
\maketitle              % typeset the header of the contribution
%
\section{Definici\'on del Problema}

En un zoológico un poco especial, se separa a los animales en dos 
hábitats diferentes de manera general. El hábitat para la reproducción 
solo acepta animales de la misma especie, mientras que el hábitat para 
la maduración admite animales de distintas especies, pero no de géneros 
distintos. El zoológico está pasando por una remodelación, ya que va 
a recibir n especies distintas de animales, cada una con un número 'a' 
de hembras y 'b' de machos, que puede ser distinto entre cada especie. 
En la remodelación se está pensando en construir salas de exhibición de 
los animales. Cada una se construirá para ser un hábitat de reproducción
 o de maduración, y cada sala podrá soportar un máximo de k animales, 
 lo cual es igual para todas las salas. Como el zoológico necesita ser 
 rentable y cada sala se cobra por separado, se quiere construir el 
 máximo número posible de salas que cumplan con los requerimientos 
 planteados y que, además, estén llenas de animales para el disfrute 
 de los visitantes. ¿Cuántas salas deberá construir el zoológico?

\section{Modelaci\'on del Problema}

Se tiene una matriz de dimensi\'on $2*n$, donde cada casilla contiene n\'umeros 
enteros. El problema consiste en hallar la cantidad m\'axima de grupos de 
tama\~{n}o $k$ que se pueden formar, donde cada elemento de un grupo es 
una unidad perteneciente a una casilla de la matriz. Adem\'as, todos los 
elementos de un grupo deben pertenecer a casillas de una misma fila o de 
una misma columna de la la matriz.

\section{Primera Aproximaci\'on}

Como primera soluci\'on se propone aplanar la matriz de entrada para 
obtener una lista de $T$ elementos, siendo $T$ la suma de todos los elementos de la matriz,
 donde en cada posici\'on hay un animal. 
Los animales son modelados como una clase \emph{Animal} que tiene dos 
atributos: \emph{gender}, fila a la que pertenece en la matriz (g\'enero),  
y \emph{specie}, columna a la que pertenece en la matriz (especie). Luego 
generar todas las permutaciones de la matriz aplanada y por cada una 
dividirla mediante \'indices en porciones de tamaño $k$, contar cuantas 
de estas porciones son grupos v\'alidos y actualizar, de ser necesario, el 
máximo;   

\subsection{Optimalidad}
Sea $S = [a_1, a_2, \dots, a_T]$ una distribución de los animales de la 
matriz aplanada, tal que la cantidad de porciones de tamaño $k$ que se pueden 
formar a partir de ella, que cumplen las restricciones del problema, sea máxima. 
Esta distribución $S$ es una permutación de los elementos de la matriz aplanada. 
El algoritmo propuesto revisa todas las permutaciones de la matriz aplanada y 
calcula la cantidad de grupos factibles, por tanto, revisa a $S$. Luego 
el algoritmo encuentra la respuesta óptima.

\subsection{Complejidad Temporal}

Aplanar la matriz tiene costo $O(T)$ donde T es la cantidad 
de animales. Computar todas las permutaciones de una lista 
de tamaño T tiene un costo $O(T!)$. La comprobación 
de cuántos grupos factibles contiene una permutación de T 
elementos tiene complejidad $O(T)$. Luego el algoritmo 
tiene complejidad $O(T + T! + T)$ y por el teorema de la 
suma, la complejidad final es $O(T!)$.

\section{Solución Greedy}

Sea $M = \sum_{i=0}^{n}matriz[0, i]$, $F = \sum_{i=0}^{n}matriz[1, i]$ y 
$T = M + F$.

La cantidad máxima de grupos que se pueden formar es $\lfloor \frac{T}{k} \rfloor$

\end{document}