\documentclass[article]{llncs}
%
\usepackage[utf8]{inputenc}
\usepackage[spanish]{babel}
\usepackage{graphicx}
% Used for displaying a sample figure. If possible, figure files should
% be included in EPS format.
%
% If you use the hyperref package, please uncomment the following line
% to display URLs in blue roman font according to Springer's eBook style:
% \renewcommand\UrlFont{\color{blue}\rmfamily}


\begin{document}
%
\title{Dise\~{n}o y An\'alisis de Algoritmos. Problema 1: El Zool\'ogico}
%
%\titlerunning{Abbreviated paper title}
% If the paper title is too long for the running head, you can set
% an abbreviated paper title here
%
\author{Jes\'us Santos Capote y Kenny Villalobos Morales}
%
\institute{Facultad de Matemática y Computación, Universidad de La Habana, La Habana, Cuba }
%
\maketitle              % typeset the header of the contribution
%
\section{Definici\'on del Problema}

En un zoológico un poco especial, a los animales se les separa en dos 
habitats diferentes de manera general. 
El habitat para la reproducción, que solo acepta animales de la misma 
especie, y el habitat para la maduración, 
que admite animales de distintas especies, pero no de géneros distintos. El zoológico está pasando por una 
remodelación ya que va a recibir n especies distintas de animales, cada una con un número 'a' de hembras y 'b' 
de machos, que puede ser distinto entre cada especie. En la remodelación se está pensando en construir salas de 
exhibición de los animales, cada una se construirá para ser un habitat de reproducción o de maduración, y cada 
sala podrá soportar un máximo de k animales, igual para todas las salas. Como el zoológico necesita ser rentable 
y cada sala se cobra por separado, se quiere construir el máximo número posible de salas que cumplan con los 
requerimientos planteados y que, además, estén llenas de animales, para el disfrute de los visitantes. Cuantas 
salas deberá construir el zoológico?

\end{document}